本研究を先行研究と比較してどのように位置づけることができるか
(どのような流れに乗っていて,何が新規な点であるか)を説明しましょう.

\subsection*{「関連研究」を書く際のアドバイス}
\begin{itemize}
    \item 「関連研究」は,本研究の位置づけを説明するためにあります.本研究との関連を述べずに紹介されている関連研究は存在意義がありませんので,関連を述べることを意識して書いてみてください.
    \item 何を関連研究として載せるかの選択は重要です.大抵は,分野,問題,解決手段に分けることが多いですが,基本的には,本研究のウリ・面白い点を際立たせるにはどのような観点で述べると良いかを考えると良いと思います.
    \item 必ずしも綺麗に時系列順に並んでいることが望ましいとは限りません.話が行ったり来たりしないように,トピックごとにまとめて書くことを意識してみてください.
\end{itemize}