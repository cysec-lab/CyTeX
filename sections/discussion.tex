本章では,評価結果を総合的に見て,知見として主張できる内容と,評価結果が示唆している内容をまとめましょう.
また,考察をもとに,本研究における課題・展望を重要な課題に絞って述べましょう.

\section{「考察・今後の展望」を書く際のアドバイス}
\begin{itemize}
    \item 「~だと思う」のように一人称視点で書くのではなく,第三者的視点で「~だといえる」「~だと示唆される」「~だと思われる」と書くようにしましょう.これはつまり,個人の感想ではなく,「結果からどう考えてもそう結論付けざるを得ない」という第三者的な目線で書くように意識して書いてみてくださいということです.
    \item 結果から強く主張できる場合は「~だといえる」,エビデンスが不十分な場合は~だと示唆される」「~だと思われる」を使いましょう.
    \item 客観的事実と著者の主張が混ざらないように気をつけましょう.
    \item 本研究の限界についても述べましょう(課題は挙げるときりがなく,述べ過ぎると具合が悪いので,重要な課題に絞って「これがまだできてなくて今後解決する必要がある」と説明する).
\end{itemize}