CyTeX の template/bachelor-2024は,2024年度の学士論文用のテンプレートである.
あくまでテンプレートであるため,章構成やディレクトリ構成等は自由に変更してよい.


\section{よく使うTeX構文}
よく使うTeXの記述例を示す.
詳しくは,TeX Wiki\cite{TeXWiki}を参照されたい.

\begin{itemize}
    \item itemize1
    \item itemize2
    \item itemize3
\end{itemize}

\begin{enumerate}
    \item enumerate1
    \item enumerate2
    \item enumerate3
\end{enumerate}

図\ref{fig:tetsutaro-logo1},図\ref{fig:tetsutaro-logo2},図\ref{fig:tetsutaro-logo3}にテツ太郎ロゴを示す.

\begin{figure}[tb]
    \centering
    \includegraphics[clip,width=5cm]{assets/introduction/tetsutaro-logo.png}
    \caption{テツ太郎ロゴ}
    \label{fig:tetsutaro-logo1}
\end{figure}


\begin{figure}[tb]
    \begin{minipage}[b]{0.45\linewidth}
        \centering
        \includegraphics[width=30mm]{assets/introduction/tetsutaro-logo.png}
        \subcaption{テツ太郎ロゴ2}
        \label{fig:tetsutaro-logo2}
    \end{minipage}
    \begin{minipage}[b]{0.45\linewidth}
        \centering
        \includegraphics[width=30mm]{assets/introduction/tetsutaro-logo.png}
        \subcaption{テツ太郎ロゴ3}
        \label{fig:tetsutaro-logo3}
    \end{minipage}
    \caption{テツ太郎ロゴ}
\end{figure}

表\ref{tab:example}に表の例を示す.

\begin{table}[tb]
    \caption{表の例}
    \label{tab:example}
    \hbox to\hsize{\hfil
        \begin{tabular}{l|lll}\hline\hline
                 & column1  & column2  & column3  \\\hline
            row1 & item 1,1 & item 2,1 & ---      \\
            row2 & ---      & item 2,2 & item 3,2 \\
            row3 & item 1,3 & item 2,3 & item 3,3 \\
            row4 & item 1,4 & item 2,4 & item 3,4 \\\hline
        \end{tabular}\hfil}
\end{table}

ソースコードの例をListing\ref{list:example}に示す.

\begin{figure}[tb]
    \begin{lstlisting}[language=Python, caption={ソースコードの例}, label={list:example}]
import numpy as np
import matplotlib.pyplot as plt

x = np.linspace(0, 2*np.pi)
y = np.sin(x)
\end{lstlisting}
\end{figure}


